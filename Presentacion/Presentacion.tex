\documentclass{beamer}

\usetheme{Madrid}

\useoutertheme{miniframes}
\useinnertheme{circles}

\definecolor{UBCblue}{rgb}{0.04706, 0.13725, 0.26667}

\usecolortheme[named=UBCblue]{structure}

\title{MOOGLE!}
\author{Sammy Sosa Justiz}

\begin{document}


\begin{frame}
  \titlepage
\end{frame}


\begin{frame}
  \frametitle{Bienvenidos}
  En este proyecto se busco crear un buscador de documentos que dada una consulta dictara mediante algoritmos antes utilizados la relevancia de los elementos de una base de datos.
\end{frame}

\begin{frame}
    \frametitle{TF-IDF}
    Este algoritmo es la base de mi busqueda y dara una manera practica de calificar a los documentos y las palabras que contienen su nombre consta de dos partes:

    \begin{itemize}
        \item Frecuencia de Término (TF)
        \item Frecuencia Inversa de Documento (IDF)
    \end{itemize}
\end{frame}

    
\begin{frame}
    \frametitle{Trabaja incluso antes de la consulta}
  	Antes de incluso saber la consulta el proyecto ya accede la base de datos y la "datifica" guardando los puntajes que reciben mediante el ya mencionado TF-IDF.Los datos guardados estan normalizados ya que no es relevante el uso de cierto tipo de caracteres.
\end{frame}

\begin{frame}
    \frametitle{Recibiendo la consulta}
    Una vez que se ha procesado la información, el programa puede realizar búsquedas.Primero buscando operadores que seguro resultaran utiles.Estos operadores hacen pequeños cambios a el tf-idf de la consulta que traen como resultado busquedas mas especificas.
\end{frame}
    
\begin{frame}
    \frametitle{Busqueda}
    El proyecto descarta o separa rapidamente Documentos puntuados para con sus respectivas puntuaciones crear una lista que sera la entrega final.
\end{frame}

\begin{frame}
    \frametitle{Vale destacar}
        \begin{itemize}
            \item La sugerencia que busca palabras alternativas que pertenecen al universo de palabras y que se asemejan a la consulta.
            \item Codigo a mi consideracion poco repetitivo y que aprovecha la informacion que acumula al maximo.
            \item El codigo es facilmente adaptable a mejorar las que estoy seguro existen.
        \end{itemize}
\end{frame}
  
\begin{frame}
    \frametitle{Operadores}
        \begin{itemize}
            \item Operador de Importancia "*"
            \item Operador de Baneo "!"
            \item Operador de Palabras Deseadas "\\ˆ"
        \end{itemize}
\end{frame}

\begin{frame}
    \frametitle{Operadores de búsqueda}
    \textbf{Operador de Importancia "*"}
    Mediante este operador a una palabra se le dara x veces importancia en la busqueda dependiendo de la cantidad de veces que se use.
\end{frame}

\begin{frame}
    \frametitle{Operadores de búsqueda}
    \textbf{Operador de Baneo "!"}
    Se usa para indicar que esta palabra no debe aparecer en ningun resultado de la busqueda.
\end{frame}
    

\begin{frame}
    \frametitle{Operadores de búsqueda}
    \textbf{Operador de aparición "\\ˆ"}
    Se usa para indicar que esta palabra es tan relevante que si un documento no la posee no debe ni siquiera ser tomado en cuenta a la hora de brindar resultados.
\end{frame}
    
    
    \begin{frame}
    \frametitle{Conclusiones}
    Creo que se logro el objetivo principal y que el codigo logrado se basta para ofrecer resultados competentes a cualquiera que llegue a usarlo.

    Invitamos a los usuarios a descargar y utilizar el programa para buscar archivos de texto en grandes conjuntos de datos.
\end{frame}

\begin{frame}
  \frametitle{¡Gracias!}
  Gracias por su atención.
\end{frame}

\end{document}